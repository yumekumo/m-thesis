\documentclass[a4j,12pt]{jreport}
%\documentclass{jreport}
\usepackage[dvipdfmx]{graphicx}
% \usepackage[dvipdfmx]{graphics}
\usepackage{amsmath,amssymb}
% \usepackage{amsmath}
%\usepackage{pxjahyper}
\usepackage{here}
\usepackage{algorithm}
\usepackage{algorithmicx}
\usepackage{algpseudocode}
\usepackage{hhline}
\usepackage[hang,small,bf]{caption}
\usepackage[subrefformat=parens]{subcaption}
\usepackage{url}
\usepackage{mathrsfs}
\usepackage{amsmath}
\usepackage{mathtools}

\makeatletter
\newenvironment{breakablealgorithm}
  {% \begin{breakablealgorithm}
   \begin{center}
     \refstepcounter{algorithm}% New algorithm
     \hrule height.8pt depth0pt \kern2pt% \@fs@pre for \@fs@ruled
     \renewcommand{\caption}[2][\relax]{% Make a new \caption
       {\raggedright\textbf{\fname@algorithm~\thealgorithm} ##2\par}%
       \ifx\relax##1\relax % #1 is \relax
         \addcontentsline{loa}{algorithm}{\protect\numberline{\thealgorithm}##2}%
       \else % #1 is not \relax
         \addcontentsline{loa}{algorithm}{\protect\numberline{\thealgorithm}##1}%
       \fi
       \kern2pt\hrule\kern2pt
     }
  }{% \end{breakablealgorithm}
     \kern2pt\hrule\relax% \@fs@post for \@fs@ruled
   \end{center}
  }
\makeatother

\captionsetup{compatibility=false}

\def\syaji{ \chapter*{謝辞} \addcontentsline{toc}{chapter}{謝辞}}
\renewcommand{\bibname}{参考文献}
\setlength{\textheight}{\paperheight}
\setlength{\topmargin}{4.6mm}
\addtolength{\topmargin}{-\headheight}
\addtolength{\topmargin}{-\headsep}
\addtolength{\topmargin}{-\headheight}
\addtolength{\textheight}{-60mm}

\setlength{\textwidth}{\paperwidth}
\setlength{\oddsidemargin}{-0.4mm}
\setlength{\evensidemargin}{-0.4mm}
\addtolength{\textwidth}{-50mm}

\begin{document}

%%%%%%%%%%%%%%%%%%%%%
% 表紙
%%%%%%%%%%%%%%%%%%%%%
\thispagestyle{empty}
\begin{center}
\begin{Large}
\vspace*{0.7cm}
{\large 中央大学大学院理工学研究科情報工学専攻\\修士論文}\\
\vspace*{2.3cm}
{\bf 選挙区割問題に対するヒューリスティクスを用いた\\ZDD構築の効率化}\\
\vspace*{0.7cm}
{\sf Efficient ZDD Construction Using Heuristics \\for the Electoral Districting Problem}\\
\vspace*{5cm}
千原 良太\\
Ryota CHIHARA\\
学籍番号\hspace*{1zw}21N8100011I\\
\vspace*{2.5cm}
指導教員\hspace*{1zw} 今堀 慎治 教授\\
\vspace*{2.5cm}
2023年3月\\
\end{Large}
\end{center}


%%%%%%%%%%%%%%%%%%%%%
% 概要
%%%%%%%%%%%%%%%%%%%%%
\newpage
\renewcommand{\baselinestretch}{1.25} \selectfont
\pagenumbering{roman}


\begin{center} {\large \bf{概 要}} \end{center}

衆議院議員選挙小選挙区制における選挙区割問題とは,
各都道府県ごとに議席数(区割数)が定められており,
市区町村からなる小地域を組み合わせて区割を構成し,
その中から最も良い区割を見つける離散最適化問題の一種である.
実際の選挙区割では,人口の偏りによる「一票の格差」が問題提起されており,
人口の格差を最小にした区割の導出が求められている.

この問題の解法として,
ゼロサプレス型二分決定グラフ(ZDD)を用いた区割列挙が知られている.
区割数や各区割の人口の上限・下限などを制約として与え,その制約から枝刈りを行うことで,
解候補を列挙することができる.
ただし,区割人口の上下限制約は,平均人口から一律に定められた格差許容定数を用いて計算し,
メモリ不足等で解が導出できない場合のみ値を変更する手法が多く取られていた.

本論文では,ヒューリスティクスを用いて区割人口の上下限制約を定め,
それを基にZDDを構築することで,
従来よりも効率的に解候補を得る手法を提案する.
また,計算機実験を行い,
従来手法よりもZDD構築における計算時間とメモリ使用量が削減できることを確認する.


\vspace{1zw} \noindent
{\bf キーワード: }離散最適化,選挙区割問題,ZDD,ヒューリスティクス.

%%%%%%%%%%%%%%%%%%%%%
% 目次
%%%%%%%%%%%%%%%%%%%%%
\tableofcontents


\newpage
\pagenumbering{arabic}

%%%%%%%%%%%%%%%%%%%%%
% 本文
%%%%%%%%%%%%%%%%%%%%%
%%%%%%%%%%%%%%%%%%%%%
% 1章
%%%%%%%%%%%%%%%%%%%%%
\chapter{はじめに} \label{chapter:1}

修士論文の書き方は修士論文要旨(年報)のテンプレートに記載されているので,参考にすること.

%%%%%%%%%%%%%%%%%%%%%
% 2章
%%%%%%%%%%%%%%%%%%%%%
\chapter{選挙区割問題} \label{chapter:2}

本章では,選挙区割問題について,国の公表資料から区割作成方針を示し,
問題の定義,数理モデルによる定式化について説明する.

\section{区割作成方針}
衆議院議員選挙の選挙区割の改定案は,
衆議院議員選挙区画定審議会によって作成される.
当審議会では,令和4年2月21日に『区割り改定案の作成方針』を公表しており,
作成方針を簡潔に述べると以下の6点となる.
\begin{enumerate}
    \item 一票の格差は2倍未満とする.
    \item 議員1人当たり人口が最も少ない県においては,各選挙区の人口をできるだけ均等にする.
    \item 改定案の作成にあたり,選挙区の区域の異動は,
    区割り基準に適合させるために必要な範囲とする.
    \item 選挙区は飛び地にしない.
    \item 選挙区を構成する市区町村は原則分割しない.
    \item 地勢,交通,人口動向などの自然的,社会的条件を総合的に考慮する.
\end{enumerate}

本研究では,項目1,2,4,5を主に考慮する.
ただし,項目2は,人口最小の県だけでなく,
全ての都道府県において各選挙区の人口をできるだけ均等にすることを目指す.
項目3は,改定前との区割の比較が研究の主旨ではないため考慮しない.
項目6は,モデル化するには複雑であるため,今回は飛び地にしないことで,
自然的,社会的条件を満たしているとみなす.

\section{問題定義}

区割作成方針をもとに問題を定義する.
まず,都道府県ごとに市区町村とその隣接関係,
各市区町村の人口を重みつきグラフ$G=(V, E, w)$で表現する.
入力は,市区町村数を$n$として,市区町村集合$V=\{v_1,...,v_n\}$, 
市区町村の隣接関係$E=\{\{v_i,v_j\}|$市区町村$v_i$と$v_j$は隣接$\}$, 
市区町村$v_i$の人口$w_i$,選挙区数$d(<n)$が与えられる.
そして出力は,$d$個の選挙区の集合$S=(S_1,...,S_d)$であり,
$S_k$は$k$番目の選挙区に属する市区町村の集合を表す.
ただし,選挙区は以下の制約を満たす必要がある.\\
\quad\textbf{制約1}:選挙区に属する市区町村から誘導される部分グラフは連結である.\\
\quad\textbf{制約2}:全ての市区町村は唯一つの選挙区に属す.\\
\quad\textbf{制約3}:選挙区は空集合ではない.

また,選挙区ごとの人口の和$P=\{P_1,...,P_d\}$ ($P_k=\sum_{v_i\in S_k}w_i\ (k=1,...,d)$)を計算し,
選挙区人口の最小値$\mathrm{min}(P)$ と最大値$\mathrm{max}(P)$を調べる.
制約を満たす選挙区割の中で,一票の格差が最小のもの,
すなわち$\frac{\mathrm{max}(P)}{\mathrm{min}(P)}$の値が最小であるものを最適な選挙区割と定める.


\section{モデル表現}
選挙区割を表す数理モデルの代表例として,集合分割型モデルを説明する.
まず,制約1から連結である市区町村の集合をブロックと名付ける.
空集合を除く全てのブロック集合を$\mathscr{B}$で表し,
ブロック集合$\mathscr{B}$から選んだ$d$個のブロックが制約2を満たすと,
実行可能な区割となる.その中で一票の格差が最小な区割を見つける.

\textbf{入力データ}:市区町村集合$V$,選挙区数$d$,
ブロック集合$\mathscr{B}$を表す行列$[b_{ij}|i=1,...,n,~j=1,...,|\mathscr{B}|]$:
市区町村$v_i$がブロック$j$の構成要素のとき$b_{ij}=1$;
そうでないとき$b_{ij}=0$,
ブロック$j$の人口$q_j=\sum_{i\in n}b_{ij}w_{i}~(j=1,...,|\mathscr{B}|)$.

\textbf{変数}:一つの選挙区の人口上限を示す変数$u$,下限を示す変数$l$,
バイナリ変数$x_j~(j=1,...,|\mathscr{B}|)$:
区割にブロック$j$を使用するとき$x_j=1$;しないとき$x_j=0$.

\textbf{定式化}:
\begin{align}
    &\mathrm{minimize} && u/l && \\
    &\mathrm{subject~to} && q_jx_j\leq u~~(j=1,...,|\mathscr{B}|) && \\
    & && \alpha(1-x_j)+q_jx_j\geq l~~(j=1,...,|\mathscr{B}|) && \\
    & && \sum_{j=1,...,|\mathscr{B}|}b_{ij}x_{j}=1 && \\
    & && \sum_{j=1,...,|\mathscr{B}|}x_j=d && \\
    & && x_j \in \{0,1\}~~(j=1,...,|\mathscr{B}|) &&
\end{align}

ここで,$\alpha$は十分大きな定数とする.
式(2.1)は一票の格差を最小化することを目的関数として示している.
式(2.2)と(2.3)は変数$u$と$l$を正しく表すために必要な制約である.
式(2.4),(2.5),(2.6)は制約2を表現している.
また,制約1と3についてはブロック集合$\mathscr{B}$から選挙区を構成しているため,
条件を満たしている.よって,上述の形で選挙区割問題を定式化することができる,
%%%%%%%%%%%%%%%%%%%%%
% 3章
%%%%%%%%%%%%%%%%%%%%%
\chapter{ZDDを用いた区割列挙手法} \label{chapter:3}

\section{概要}

\section{ゼロサプレス型二分決定グラフ}

\section{フロンティア法}

\section{区割列挙アルゴリズム}

\subsection{人口制約なしの場合}

\subsection{人口制約ありの場合}

%%%%%%%%%%%%%%%%%%%%%
% 4章
%%%%%%%%%%%%%%%%%%%%%
\chapter{ヒューリスティクスを用いた手法} \label{chapter:4}

本章では,選挙区割問題を解くヒューリスティクスを作り,
その解から得られた選挙区の人口上限と下限を用いて,
ZDD構築を効率化する手法について提案する.

\section{概要}

\ref{chapter:3}章の人口制約付きZDD構築アルゴリズムでは,
パラメータとして部分グラフの重み上限$U$,下限$L$を用いた.
許容格差定数$r$では,$r$の値によって,
$U, L$の範囲が必要以上に広くなることや
$U, L$の範囲に解が一つも存在しないことがあり得る.
既存手法では,$r$の値を都道府県によって手動で調整するものがほとんどである.
そこで本研究では,選挙区割問題をヒューリスティクスで解き,
解の中で最大人口の選挙区の重みを$U$,最小人口の選挙区の重みを$L$
として定義する手法を提案する.
$U, L$が最適解に近くなればなるほど,ZDDの構築時に,
最適解でない解候補の枝刈りの回数が多くなる.
その結果,従来手法に比べて
メモリの使用量の減少及び計算時間の短縮が期待できる.

\begin{figure}[htbp]
  \centering
  \includegraphics[scale=0.39]{img/heuristics.png}
  \caption{ヒューリスティクスを用いた提案手法}
  \label{heuristics}
\end{figure}

ヒューリスティクスでは,2.2節で定義した問題から.
評価に用いるスコアを一票の格差$\frac{\mathrm{max}(P)}{\mathrm{min}(P)}$として,
これを最小化することを目指す.

提案手法の全体像を図\ref{heuristics}に示す.
始めに選挙区割問題の初期解を生成し,次に一定の確率で
シフト近傍,スワップ近傍,2連鎖シフト近傍のいずれかを選択する.
選択した近傍操作を行って近傍解を生成し,
近傍解が選挙区割の制約(選挙区が連結であるか)を満たすか判定をする.
制約を満たしている場合には,スコアを計算し,
焼きなまし法により解の遷移を行う.
これを一定時間繰り返すことで,解を得ることができる.
その解から,$U = \mathrm{max}(P), L = \mathrm{min}(P)$
を定義し,それらを用いてZDDを構築する.
次節から,初期解生成や近傍探索のアルゴリズムについて詳細に説明する.

\section{初期解生成} \label{section:4.2}

初期解は,市区町村を表現したグラフから
選挙区を構成する部分グラフの$d$個の根を定め,
それぞれの根から幅優先探索(BFS)を行い,探索した頂点を
部分グラフに含めることで構築する.
初期解生成の擬似コードを\textbf{Algorithm 4}で示す.

\begin{breakablealgorithm}
  \caption{MakeInitialSolution}
  \label{make_initial_solution}
  \begin{algorithmic}[1]
    \Require $n, d, ave, adj\_list, w$
    \Ensure $group$
    \State $group \gets [-1] * n$
    \State $P \gets [0] * d$
    \State 頂点重みが最も大きいものから順に$d$個の頂点番号を配列$root$に保存する.
    \State $root$.reverse()
    \For {$i = 1,\ldots,d$}
      \State $queue \gets \phi $
      \State $group[root[i]] \gets i$
      \State $P[i] \gets P[i] + w[nv]$
      \State $queue$.enqueue$(root[i])$
      \While{$queue$.size() $\neq 0$ and $P[i] < ave$}
        \State $v \gets queue$.dequeue()
        \ForAll {$nv \in adj\_list[v]$}
          \If {$P[i] \geq ave$}
            \State break
          \ElsIf {$group[nv] = -1$}
            \State $group[nv] \gets i$
            \State $P[i] \gets P[i] + w[nv]$
            \State $queue$.enqueue($nv$)
          \EndIf
        \EndFor
      \EndWhile
    \EndFor
    \While{$group$の要素に$-1$が含まれる}
      \For {$vi = 1,\ldots,n$}
        \If{$group[vi]=-1$}
          \ForAll {$nv \in adj\_list[v]$}
            \If{$group[nv] \neq -1$}
              \State $group[vi] \gets group[nv]$
              \State break
            \EndIf
          \EndFor
        \EndIf
      \EndFor
    \EndWhile
    \State \textbf{return }$group$
  \end{algorithmic}
\end{breakablealgorithm}

\textbf{Algorithm 4} MakeInitialSolution の入力と出力については以下の通りである.\\
\textbf{入力}
\begin{itemize}
  \item $n$:グラフの頂点$v$の個数
  \item $d$:区割(グラフ分割)数
  \item $ave$:一選挙区の平均人口
  \item $adj\_list$:グラフの隣接リスト
  \item $w$:グラフの頂点重み
\end{itemize}
\textbf{出力}
\begin{itemize}
  \item $group$:頂点番号を添字とした配列で,
    選挙区(部分グラフ)のラベル(値は$1,\ldots,d$)が保存されている.初期値は$-1$.
\end{itemize}

2行目の$P$は部分グラフごとの重み和を保存する変数である.
配列$root$は,BFSの始点となる$d$個の頂点を含める.
3行目にもある通り,$root$は頂点重みが最も大きいものから順に指定する.
4行目で$root$を逆順にするが,これは$root$の頂点重みが小さいものから
探索を行うためである.
BFSでは,キューを用いて探索を行い,
探索した頂点を属する部分グラフに加えていく(5-22行目).
ラベル$i$の探索では,頂点$nv$を追加する際に
$group[nv]$に$i$を代入し,部分グラフの重み和$P[i]$に
$w[nv]$を加算する.
そして,重み和$P[i]$が平均重み$ave$以上になると,
その部分グラフでの探索を中断する(10,13行目).
これを全ての根$root[1],\ldots,root[d]$に対して行う.
全てのBFSを終えた後,ラベルが付かなかった($group$の値が$-1$の)
頂点に対して隣接頂点を参照し,そこからラベルを割り振る
処理を行なっている(23-34行目).
最後に$group$を出力し,これを初期解とする.

\section{近傍操作}

近傍操作とは,現在の解から少し形を変えた近傍解を生成する操作のことである.
選挙区割問題における近傍操作では,次の3つを定義する.
\begin{itemize}
  \item シフト近傍
  \item スワップ近傍
  \item 2連鎖シフト近傍
\end{itemize}

3つの近傍操作は確率を用いてその都度ランダムに選択する.
近傍解は,部分グラフが連結でない可能性があるため,
その都度BFSを用いて連結であるかの判定を行う.
近傍解が連結かつ解の改善が見込める場合には,
近傍解を新たな解として採用する.
これを一定時間繰り返し,解の改善を行う.

\subsection{シフト近傍}

シフト近傍は,一頂点のラベルを変更する近傍操作である.
シフト近傍の例として,図\ref{shift-neighbor}を用いて説明する.
青色の頂点集合からなる部分グラフを$\mathcal{G}_1$,
赤色の頂点集合からなる部分グラフを$\mathcal{G}_2$とする.
$\mathcal{G}_1$には頂点$v_1$が含まれており,$v_1$は,
別の部分グラフ($\mathcal{G}_2$)の頂点$v_2$と隣接している.
シフト近傍では,頂点$v_1$のラベル$group[v_1]$を
頂点$v_2$のラベル$group[v_2]$に代入することで,
$v_2$を$\mathcal{G}_2$から$\mathcal{G}_1$に含めることができる.
このようにして出来た新たな$group$を近傍解とする.

\begin{figure}[htbp]
  \centering
  \includegraphics[scale=0.2]{img/shift-neighbor.png}
  \caption{シフト近傍}
  \label{shift-neighbor}
\end{figure}

\subsection{スワップ近傍}

\begin{figure}[htbp]
  \centering
  \includegraphics[scale=0.2]{img/swap-neighbor.png}
  \caption{スワップ近傍}
  \label{swap-neighbor}
\end{figure}

\subsection{2連鎖シフト近傍}

\begin{figure}[htbp]
  \centering
  \includegraphics[scale=0.2]{img/chain-neighbor.png}
  \caption{2連鎖シフト近傍}
  \label{chain-neighbor}
\end{figure}

\section{焼きなまし法}


%%%%%%%%%%%%%%%%%%%%%
% 5章
%%%%%%%%%%%%%%%%%%%%%
\chapter{計算機実験} \label{chapter:5}

\section{概要}
本章では,提案手法であるヒューリスティクス解を用いたZDD構築の
計算機実験を行い,既存手法の人口制約のないZDD構築手法,
人口制約ありで許容格差定数を用いたZDD構築手法との比較をし,
提案手法の評価及び考察を行う.

\section{実験環境}
実験環境は,次の通りである.

\begin{itemize}
  \item OS:Ubuntu 20.04.5 LTS
  \item CPU:Intel Xeon E5-2687W v4(3.0GHz)
  \item メモリ:512GB
\end{itemize}

プログラムはC++17によって実装し,gccを用いて -O3, -march=native
オプションを付与してコンパイルを行った.また,ライブラリとして,
SAPPOROBDD,TdZdd を利用した.

\section{入力データ}
入力データとして,国土交通省が公開している「国土数値情報 行政区域データ」と
令和2年国勢調査結果の「人口等基本集計」を利用し,
市区町村を頂点,隣接関係を辺,人口を頂点重みとした
グラフを作成した.
本論文ではいくつかの都道府県のインスタンスを抜粋して掲載する.
入力データについて,各インスタンスのパラメータを表\ref{input_data}にまとめた.

\begin{table}[htbp]
  \caption{入力データ}
  \label{input_data}
  \centering
  \begin{tabular}{l|rrr}
    \hline
    Name & $|V|$ & $|E|$ & $d$ \\
    \hline \hline
    $G_1$(Aomori) & 40 & 84 & 3 \\
    $G_2$(Miyagi) & 39 & 86 & 5 \\
    $G_3$(Yamagata) & 35 & 85 & 3 \\
    $G_4$(Fukushima) & 59 & 144 & 4 \\
    $G_5$(Ibaraki) & 44 & 94 & 7 \\
    $G_6$(Nagano) & 77 & 187 & 5 \\
    $G_7$(Aichi) & 69 & 173 & 16 \\
    $G_8$(Osaka) & 72 & 168 & 19 \\
    \hline
  \end{tabular}
\end{table}

表\ref{input_data}の$|V|$は頂点数,$|E|$は辺数,$d$は分割数を
表している.
また,頂点の各重みについて分布図を図\ref{w_dist}にまとめた.

\begin{figure}[bp]
  \begin{tabular}{cc}
    \begin{minipage}[t]{0.45\hsize}
      \centering
      \includegraphics[keepaspectratio, scale=0.5]{img/g1.png}
      \subcaption{$G_1$}
      \label{g1}
    \end{minipage} &
    \begin{minipage}[t]{0.45\hsize}
      \centering
      \includegraphics[keepaspectratio, scale=0.5]{img/g2.png}
      \subcaption{$G_2$}
      \label{g2}
    \end{minipage} \\

    \begin{minipage}[t]{0.45\hsize}
      \centering
      \includegraphics[keepaspectratio, scale=0.5]{img/g3.png}
      \subcaption{$G_3$}
      \label{g3}
    \end{minipage} &
    \begin{minipage}[t]{0.45\hsize}
      \centering
      \includegraphics[keepaspectratio, scale=0.5]{img/g4.png}
      \subcaption{$G_4$}
      \label{g4}
    \end{minipage} \\
    \begin{minipage}[t]{0.45\hsize}
      \centering
      \includegraphics[keepaspectratio, scale=0.5]{img/g5.png}
      \subcaption{$G_5$}
      \label{g5}
    \end{minipage} &
    \begin{minipage}[t]{0.45\hsize}
      \centering
      \includegraphics[keepaspectratio, scale=0.5]{img/g6.png}
      \subcaption{$G_6$}
      \label{g6}
    \end{minipage} \\

    \begin{minipage}[t]{0.45\hsize}
      \centering
      \includegraphics[keepaspectratio, scale=0.5]{img/g7.png}
      \subcaption{$G_7$}
      \label{g7}
    \end{minipage} &
    \begin{minipage}[t]{0.45\hsize}
      \centering
      \includegraphics[keepaspectratio, scale=0.5]{img/g8.png}
      \subcaption{$G_8$}
      \label{g8}
    \end{minipage}
  \end{tabular}
  \caption{頂点の重み分布}
  \label{w_dist}
\end{figure}


\section{実験結果}

\subsection{人口制約なし}

3.3.1項にて述べた人口制約のない区割列挙の結果は,
表\ref{out_normal}の通りである.

\begin{table}[htbp]
  \caption{人口制約なし区割列挙}
  \label{out_normal}
  \centering
  \begin{tabular}{l||r|r|r|r}
    \hline
    Name & node & solve & time(sec) & memory(MB) \\
    \hline \hline
    $G_1$(Aomori) & 5196 & 10452641 & 0.02 & 6 \\
    $G_2$(Miyagi) & 2891 & 1.98E+10 & 0.01 & 4 \\
    $G_3$(Yamagata) & 2672 & 490516246 & 0.01 & 4 \\
    $G_4$(Fukushima) & 233446 & 1.51E+14 & 1108.69 & 16859 \\
    $G_5$(Ibaraki) & 10757 & 3.24E+13 & 0.02 & 6 \\
    $G_6$(Nagano) & 48612 & 3.82E+17 & 1.48 & 472 \\
    $G_7$(Aichi) & 1145106 & 3.02E+29 & 2.41 & 395 \\
    $G_8$(Osaka) & 955147 & 1.73E+30 & 0.5 & 92 \\
    \hline
  \end{tabular}
\end{table}


\subsection{人口制約あり:許容格差定数を用いる場合}

\begin{table}[htbp]
  \caption{$r=1.4$とした区割列挙}
  \label{out_r}
  \centering
  \begin{tabular}{l||r|r||r|r|r|r}
    \hline
    Name & $L$ & $U$ & node & solve & time(sec) & memory(MB) \\
    \hline \hline
    $G_1$(Aomori) & 325785 & 509759 & 23749 & 668154 & 2.25 & 405 \\
    $G_2$(Miyagi) & 348787 & 596814 & 6581 & 40106 & 3.38 & 652 \\
    $G_3$(Yamagata) & 281059 & 439776 & 319171 & 7493473 & 33.81 & 6702 \\
    $G_4$(Fukushima) & 585129 & 353649 & N/A & N/A & N/A & N/A \\
    $G_5$(Ibaraki) & 305000 & 542408 & 1077156 & 36745326 & 212.75 & 30690 \\
    $G_6$(Nagano) & 310304 & 530966 & N/A & N/A & N/A & N/A \\
    $G_7$(Aichi) & 342837 & 643865 & N/A & N/A & N/A & N/A \\
    $G_8$(Osaka) & 337316 & 637772 & N/A & N/A & N/A & N/A \\
    \hline
  \end{tabular}
\end{table}


\subsection{人口制約あり:ヒューリスティクスの結果を用いる場合}

\begin{table}[htbp]
  \caption{ヒューリスティクスを用いた区割列挙}
  \label{out_h}
  \centering
  \begin{tabular}{l||r|r||r|r|r|r}
    \hline
    Name & $L$ & $U$ & node & solve & time(sec) & memory(MB) \\
    \hline \hline
    $G_1$(Aomori) & 226194 & 555698 & 34609 & 2001248 & 2.77 & 460 \\
    $G_2$(Miyagi) & 451162 & 467561 & 55 & 2 & 0.01 & 4 \\
    $G_3$(Yamagata) & 355396 & 356505 & 4416 & 541 & 0.08 & 19 \\
    $G_4$(Fukushima) & 459096 & 460480 & N/A & N/A & N/A & N/A \\
    $G_5$(Ibaraki) & 391937 & 419212 & 1340 & 390 & 0.02 & 6 \\
    $G_6$(Nagano) & 408772 & 410752 & 10320 & 17657 & 2.3 & 471 \\
    $G_7$(Aichi) & 359399 & 553700 & 1847085 & 1.29E+14 & 89.02 & 12607 \\
    $G_8$(Osaka) & 424530 & 569011 & N/A & N/A & N/A & N/A \\
    \hline
  \end{tabular}
\end{table}


\section{考察}
%%%%%%%%%%%%%%%%%%%%%
% 6章
%%%%%%%%%%%%%%%%%%%%%
\chapter{計算機実験} \label{chapter:6}


%謝辞
\syaji
\par
本研究を進めるにあたり,大変多くのご指導,ご助言を頂いた
中央大学大学院理工学研究科情報工学専攻の今堀慎治教授,
ZDDの実装にあたりライブラリの提供やご相談に快く応じて頂いた
京都大学大学院情報学研究科通信情報システム専攻の川原純准教授に深く感謝いたします.
また,多大なるご助言,ご協力を頂いた今堀研究室の皆様には大変お世話になりました.
心から感謝いたします.

最後に,大学4年間に加え,大学院に2年間通わせていただいた両親に深く感謝いたします.

\chapter*{関連発表}
\addcontentsline{toc}{chapter}{関連発表}
\begin{enumerate}
  \item 千原良太,今堀慎治:選挙区割問題に対するヒューリスティクスを用いたZDD構築の効率化,
  日本オペレーションズ・リサーチ学会 2023年春季研究発表会,2023年3月8日
\end{enumerate}

%参考文献
\begin{thebibliography}{99}
\addcontentsline{toc}{chapter}{参考文献}

\bibitem{ichimori}
一森哲夫:議席配分の数理-選挙制度に潜む200年の数学-,近代科学社 (2022).

\bibitem{nemoto}
根本俊男,堀田敬介:区割画定問題のモデル化と最適区割の導出,オペレーションズ・リサーチ,
vol.~48,no.~4,pp.~300-306 (2003).

\bibitem{kawahara}
Kawahara, J. et al.: Generating All Patterns of Graph Partition Within a Disparity Bound,
WALCOM, pp. 119-131 (2017).

\bibitem{minato}
Minato, S.: Zero-suppressed BDDs for set manipulation in combinatorial problems,
Proceedings of the 30th international Design Automation Conference,
LNCS-10167, pp.~272-277 (1993).

\bibitem{minato_or}
湊真一:BDD/ZDDを用いたグラフ列挙索引化技法,オペレーションズ・リサーチ,
vol.~57,no.~11,pp.~597-603 (2012).

\bibitem{sekine}
Sekine, K., Imai, H. Tani, S.: Computing the Tutte polynomial of a graph of moderate size,
Proceedings of the 6th International Symposium on Algorithms and Computation (ISAAC),
LNCS-1004, pp.~224-233 (1995).

\bibitem{umetani}
梅谷俊治:しっかり学ぶ数理最適化 モデルからアルゴリズムまで,講談社 (2020).

\bibitem{iwashita}
Iwashita, H. and Minato, S.: TCS Technical Report Efficient Top-Down
ZDD Construction Techniques Using Recursive Specifications,
TCS Technical report (2013).

\bibitem{yamazaki}
山崎宏紀:部分グラフ列挙問題に対する反復的トップダウン ZDD 構築手法の研究,
京都大学大学院情報学研究科 修士課程通信情報システム専攻 修士論文 (2022).

\end{thebibliography}

%関連論文, 仕様はthebibliographyと同一. 
%\begin{therelatedreference}{99}
%\end{therelatedreference}

\end{document}