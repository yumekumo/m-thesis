%%%%%%%%%%%%%%%%%%%%%
% 1章
%%%%%%%%%%%%%%%%%%%%%
\chapter{はじめに} \label{chapter:1}
日本の衆議院議員選挙における小選挙区制の区割は,
総定数から各都道府県に何議席を割り当てるかを決める定数配分問題と
都道府県内で,割り当てられた議席数分の選挙区を市区町村を組合せて構築する
区割画定問題を解くことによって,定めることができる.

定数配分問題は,法学,公共政策学,数理情報学などの様々な観点から
取り組まれており,過去200年以上にわたり多くの手法が提案されている.
2022年12月28日には,公職選挙法の一部を改正する法律(区割り改定法)が施行され,
衆議院小選挙区選出議員の選挙区について「アダムズ方式」を用いた議席の配分が行われた\cite{ichimori}.
アダムズ方式はアメリカ6代目大統領ジョン・アダムズが考案したとされており,
簡単に説明すると
「各都道府県の人口をある自然数で割った商の小数点以下を切り上げた数を,
その都道府県の議席数とする」手法である.
この手法は,一票の格差の是正には効果的とされているが,
「アラバマ・パラドックス」と呼ばれる改訂時に議席総数が増加した際に,
ある地区では配分される議席数が改訂前より減る現象が起こる場合がある.
また,過去に提案された手法を比較したときに,一票の格差がほぼ等しい場合でも,
各都道府県の人口が多い方が有利な手法,少ない方が有利な手法といった
差が現れ,どの手法も一長一短であることから,
選挙制度の意義等も踏まえつつ議論する必要がある.
本稿では,定数配分問題については主に扱わず,2021年に行われた第49回衆議院議員選挙の
定数配分をそのまま利用する.

区割画定問題は,都道府県内の選挙区の組合せが市区町村数の指数通り存在し,
NP困難であるとして,20世紀末までは最適性の保障のない解の導出の研究が主であった.
しかし,2003年に根本・堀田が数理モデルによる定式化を提案\cite{nemoto}して以降,
数理的な観点から多くの研究が取り組まれており,
厳密解を導出するための手法がいくつか提案されている.
その中の手法の一つとして,ゼロサプレス型二分決定木(ZDD)を用いた
区割列挙があり,フロンティア法によってトップダウンにZDDを構築することで,
高速に選挙区割を求めることが可能となっている.
ただし,いくつかの都道府県においては計算機のメモリ不足により解を導出することが
困難である.

本研究では,区割画定問題(以下「選挙区割問題」と称する)における
ZDDを用いた区割列挙について扱い,
ヒューリスティクスを用いて効率的に区割列挙を行う手法を提案する.
本稿の第2章では選挙区割問題について定義し,
数理モデルによる定式化を説明する.
第3章ではZDDとフロンティア法,
それを用いた区割列挙アルゴリズムについて詳しく述べる.
第4章では,第3章で説明したアルゴリズムとヒューリスティクスを組合せることで
ZDD構築を効率化する手法を提案する.
そして,第6章で計算機実験の結果とその考察を示し,
第7章で結論と今後の課題について述べる.
