%%%%%%%%%%%%%%%%%%%%%
% 2章
%%%%%%%%%%%%%%%%%%%%%
\chapter{修士論文の書き方} \label{chapter:2}


\section{図と表の例} \label{section:figure_table}

図・表には通し番号と見出し(caption)を付け,本文中で当該の図・表に言及する.また,単位や目盛を正確に記す.
図のタイトルは図の下に,表のタイトルは表の上に書く.

例を図\ref{fig:logo}と表\ref{tab:results}に示す.
第\ref{chapter:2}章の図には図2.1, 図2.2, 図2.3,…のように番号が振られ,
第\ref{chapter:2}章の表には表2.1, 表2.2, 表2.3,…のように,図とは独立に番号が振られる.


\begin{figure}[b]
    \centering
    \includegraphics[scale=0.5]{logo_color.png}
    \caption{情報工学科のロゴ}
    \label{fig:logo}
  \end{figure}


\begin {table}[t]
    \centering
  \caption{表のタイトル}
  \label{tab:results}
  \begin {tabular}{ccc} \hline
     項目1 & 項目2 & 項目3 \\ \hline
    データ1 & データ2 & データ3 \\
    データ1 & データ2 & データ3 \\
    データ1 & データ2 & データ3 \\ \hline
  \end {tabular}
\end {table}


\section{参考文献の書き方}

一例として,和文の著書\cite{suetake},和文の論文誌\cite{kusano},英文の編書\cite{fuortes},
英文の論文誌\cite{rice},国際会議\cite{guibas},修士論文\cite{chudai},電子雑誌\cite{iwama},Webページ\cite{IPSJ}を,
2ページの参考文献の節に載せる.{\em 参考文献には信頼性が高く,後世に残るものを載せるように注意せよ.}

書くべき情報は以下のとおりである.
\begin{itemize}
\item 和文の著書: 著者,書名,シリーズ名(あれば),発行所,都市,年.
\item 和文の論文誌: 著者,題名,誌名,巻,号,頁,年.
\item 英文の編書: 編者,書名,発行所,都市,年.
\item 英文の論文誌: 著者,題名,誌名,巻,号,頁,年.
\item 国際会議: 著者,題名,予稿集名,都市,コード等,頁,年.
\item 修士論文: 著者,題名,機関名,年.
\item 電子雑誌: 著者,題名,誌名,巻,号,頁(オンライン),DOI,西暦年.
\item Webページ: 著者,Webページの題名,Webサイトの名称(オンライン)(ただし,著者と同じ場合は省略可),入手先〈URL〉(参照日付).
\end{itemize}
英語の文献はすべて半角文字で書く.参考文献には本文で引用した文献のみ載せる.
情報処理学会の論文誌の原稿執筆案内\cite{IPSJ}も参考になる.

通し番号は,引用順または著者名のアルファベット順に付ける.
文献の引用のしかたは分野ごとに異なるので,{\em 自己流では書かず,当該分野の論文誌などを参考にする}こと.
