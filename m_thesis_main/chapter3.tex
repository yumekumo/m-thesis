%%%%%%%%%%%%%%%%%%%%%
% 3章
%%%%%%%%%%%%%%%%%%%%%
\chapter{ZDDを用いた区割列挙手法} \label{chapter:3}

\section{概要}

\section{ゼロサプレス型二分決定グラフ}
ゼロサプレス型二分決定グラフ(ZDD)は,組合せ集合を表す非巡回有向グラフで,
1993年に湊真一によって考案された\cite{minato}.
組合せ集合とは「$n$個のアイテムから任意個を選ぶ組合せ」を要素とする集合である.
$n$個のアイテムから任意個を選ぶ組合せは$2^n$通りあるので,
その組合せ集合は,$2^{2^n}$通り存在する.
例えば,$a,b,c$の3つの要素から組合せ集合を作る場合,
$\{ab, ac, c\}, \{a\}, \{\lambda, abc\}, \phi$などが挙げられる
($\lambda$は空の組合せ要素,$\phi$は空の集合を表す).

\begin{figure}[htbp]
  \begin{minipage}[b]{0.48\hsize}
    \centering
    \includegraphics[scale=0.25]{img/binary_graph.png}
    \subcaption{場合分け二分木}
    \label{binary_graph}
  \end{minipage}
  \begin{minipage}[b]{0.48\hsize}
    \centering
    \includegraphics[scale=0.25]{img/zdd.png}
    \subcaption{ZDD}
    \label{zdd_graph}
  \end{minipage}
  \caption{組合せ集合$S=\{ab,ac,c\}$のグラフ表現}\label{combined_set}
\end{figure}

このような指数的な数の集合は,ZDDを用いることで効率的に表現することができる.
図\ref{combined_set}は,組合せ集合$S=\{ab,ac,c\}$を
場合分け二分木とZDDの両方で表した例である.
この二つのグラフは,各節点に要素を表すラベルが割り当てられていて,
0と1のラベルが付与された2種類の枝が分岐している.
そして葉には0または1の値が記入されている.
以降は0のラベルをもつ枝を0-枝(破線),1のラベルをもつ枝を1-枝(実線)とし,
また,0の値が記入された葉を0-終端(0-terminal),
1の値が記入された葉を1-終端(1-terminal)と呼ぶ.
これらのグラフでは,1-枝と0-枝はその接点の要素を選ぶかどうかの場合分けを表し,
葉の値はその葉に対応する組合せが集合に属するかを示している.

場合分け二分木とZDDを比較すると,
ZDDは場合分け二分木で集合の表現に不要な頂点と枝を削除,圧縮していることがわかる.
場合分け二分木からZDDを構築するには,次の2つの規則を可能な限り適用する.

\begin{description}
  \item[冗長頂点の削除] 1-枝が0-終端を指している場合に,この節点を取り除き,
  0-枝の行き先に直結させる(図\ref{delete_zdd}).
  \item[等価節点の共有] 等価な節点(ラベルが同じで,0-枝同士,1-枝同士の行き先が同じ)
  を共有する(図\ref{share_zdd}).
\end{description}

\begin{figure}[htbp]
  \begin{minipage}[b]{0.48\hsize}
    \centering
    \includegraphics[scale=0.25]{img/delete_zdd.png}
    \subcaption{冗長節点の削除}
    \label{delete_zdd}
  \end{minipage}
  \begin{minipage}[b]{0.48\hsize}
    \centering
    \includegraphics[scale=0.25]{img/share_zdd.png}
    \subcaption{等価節点の共有}
    \label{share_zdd}
  \end{minipage}
  \caption{ZDDの圧縮規則}\label{comp_rule}
\end{figure}

また,ZDDは組合せ集合をコンパクトに表現できるだけでなく,ZDD同士で集合演算が定義されており,
共通集合や差集合などを表すZDDを高速に得ることができる.
これを利用することで,
愚直に場合分け二分木を作るよりも高速にZDDを構築するボトムアップな構築手法\cite{minato}
が知られている.
ただし,アイテム数に対して指数的な個数の組合せを生成するような集合演算を行う場合は注意が必要であり,
集合演算の順序によっては圧縮がうまく機能せず,ZDDのサイズが指数的に増大する恐れがある.
このような場合は,計算を行う前に圧縮度を推定することは一般的に困難である.

一方,根から順にZDDを作成する「フロンティア法」と呼ばれるトップダウンな構築手法が存在する.
これについては,次節以降で説明する.

\section{フロンティア法}


\section{区割列挙アルゴリズム}

\subsection{人口制約なしの場合}

\subsection{人口制約ありの場合}
