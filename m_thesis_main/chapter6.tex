%%%%%%%%%%%%%%%%%%%%%
% 6章
%%%%%%%%%%%%%%%%%%%%%
\chapter{おわりに} \label{chapter:6}
\section{まとめ}
本研究では,選挙区割問題に対して,
ヒューリスティクスとして焼きなまし法を実装し,
得られた解を利用して
効率的にZDDを構築する手法を提案した.
許容格差$r$で制約を決める既存手法では,
人の手でZDDが構築可能な値を探索しなければいけなかったが,
提案手法によりその作業を不要にし,さらにZDD構築において
計算時間と消費メモリを大幅に削減する効果が得られることを
確認した.

\section{今後の課題}
ヒューリスティクスの解は,一部のインスタンスで
最適解から大きく離れたものであったため,
アルゴリズムを改善する必要がある.
また,今回用いたフロンティア法では,ヒューリスティクスを改良しても
解の得られない都道府県が存在するため,
全ての都道府県を列挙するためには,別のZDD構築アルゴリズムが必要となる.
関連する研究として,ZDDを反復的にトップダウンで構築する
サブセッティング法\cite{iwashita}を
選挙区割問題に適用した事例\cite{yamazaki}が存在する.
サブセッティング法にヒューリスティクスで求めた重み上下限制約
を組合せることで,
全ての都道府県について選挙区をより高速に列挙できる可能性が考えられる.
